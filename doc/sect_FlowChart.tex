\section{Flow Chart}
\label{sect:FlowChart}

\noindent This Section is devoted to the structure of the \FFW and illustrates to flow of the framework. In Figure~\ref{sect:FlowChart.fig.FFW-flowchart} one can see this structure in a flow chart.

\begin{figure}[ht!]
\footnotesize
\begin{equation*}
\xymatrix@-1eM{
    *+<2ex>[F]{ \phantom{I}\text{script}\phantom{I} }           \ar@{=>}[r]
&   *+<2ex>[F]{ \phantom{I}\code{initFFW}\phantom{I} }          \ar@{=>}[r]
&   *+<2ex>[F]{ \phantom{I}\code{run}\phantom{I} }              \ar@{=>}[r]
&   *+<2ex>[F]{ \phantom{I}\code{computeSolution}\phantom{I} }  \ar@{=>}[r] \ar@{->}[d]^{\text{loop}}
&   *+<2ex>[F]{ \phantom{I}\text{output}\phantom{I} }
\\
&&&  *+[F-:<3pt>]{ \code{afem} }            \ar@{->}[d]
\\
&&&  *+[F-:<3pt>]{ \code{mark} }            \ar@{->}[d]
\\
&&& *+[F-:<3pt>]{ \code{refine} }          \ar@{->}[d]
\\
&&& *+[F-:<3pt>]{ \code{enumerate} }       \ar@{.>}[r] \ar@{->}[d]
&   *+[F-:<3pt>]{ \code{genericEnumerate} }
\\
&&& *+[F-:<3pt>]{ \code{createLinSys} }    \ar@{->}[d]
\\
&&& *+[F-:<3pt>]{ \code{solve} }           \ar@{->}[d]
\\
&&& *+[F-:<3pt>]{ \code{postProc} }        \ar@{->}[d]
\\
&&& *+[F-:<3pt>]{ \code{estimate} }        \ar@{->}'l[u] '[uu] [uu]
}
\end{equation*}
\normalsize \caption{Flow chart of the \FFW.}\label{sect:FlowChart.fig.FFW-flowchart}
\end{figure}

\begin{longtable}{p{0.25\textwidth}p{0.65\textwidth}}
function name & description\\\hline\\[-1ex]

\text{script} & control script, examples are \code{start\-Scalar}
                and \code{start\-Elasticity}\\
\code{initFFW} & sets paths, loads default parameters,
                 sets supplied parameters and loads problem definition and geometry\\
\code{run} & calls computeSolution and creates method specific
             function handles, e.g. a function handle to evaluate
             the discrete solution\\
\code{computeSolution} & calls \code{afem} as long as some abort criteria defined
                         in the configuration file, e.g., \code{maxNrDoF},
                         is not fulfilled and supplies method type specific
                         function handles, e.g. a function to evaluate
                         the energy error\\
\code{afem} & calls \code{mark}, \code{refine}, \code{enumerate},
              \code{createLinSys}, \code{solve}, \code{postProc} and
              \code{estimate}\\
\code{mark} & marks edges and triangles for refinement based on an error
              estimator (\code{bulk}, \code{max}) with additional edges to
              maintain shape regularity, or marks all edges (\code{uniform})\\
\code{refine} & refines the list of edges supplied by mark\\
\code{enumerate} & calls \code{genericEnumerate} and computes method
                   specific data structures from the triangulation,
                   e.g., the number of degrees of freedom\\
\code{genericEnumerate} & creates useful data structures from the current
                          triangulation, e.g., normals, tangents, areas\\
\code{createLinSys} & assembles the global linear system of equations $Ax=b$\\
\code{solve} & computes $x$\\
\code{postProc} & post processing of the solution, e.g., separating Lagrange
                  multipliers form the discrete solution\\
\code{estimate} & locally estimates the error based on the discrete solution\\
\text{output} & user specific data evaluation
\end{longtable}
