%%=============================
%% Used packages
%%=============================

%%Document Style
\usepackage{a4}
\usepackage[pdftex,pagebackref,colorlinks]{hyperref}
%\usepackage{url}
%%Font Coding
\usepackage[T1]{fontenc}
%%Text Coding
\usepackage[ansinew]{inputenc}
%%Language
\usepackage[english]{babel}
%%AMS Packages
\usepackage{amssymb}
\usepackage{amsthm}
%Graphics
\usepackage[pdftex]{graphicx}
\usepackage[pdftex]{color}
\usepackage{pict2e}
\usepackage[all]{xy}
%Special Environments
\usepackage[vflt]{floatflt}
\usepackage{enumerate}
\usepackage{longtable}
\usepackage{fancyvrb}


%%=============================
%% Defined Environments
%%=============================
\theoremstyle{plain}
\newtheorem{theorem}{Theorem}[section]
\newtheorem{proposition}[theorem]{Proposition}
\newtheorem{lemma}[theorem]{Lemma}
\newtheorem{corollary}[theorem]{Corollary}
\theoremstyle{definition}
\newtheorem{definition}[theorem]{Definition}
\newtheorem{exercise}{Exercise}
\theoremstyle{remark}
\newtheorem*{remark}{Remark}
\newtheorem*{remarks}{Remarks}
\newtheorem*{example}{Example}
\newtheorem*{warning}{WARNING}
\newtheorem*{algorithm}{\textsc{Algorithm}}


%for non fragile code, e.g. without _\%
\def\code{\texttt}
%for fragile inline code use: \verb""
%for paths use \path{}

%paragraphed code
\DefineVerbatimEnvironment{pcode}{Verbatim}{fontsize=\footnotesize,frame=lines,framesep=2ex,xleftmargin=1.5em}
%paragraphed code with line numbering (it is possible to give a line a label with: \label{..} after the line as usual)
\DefineVerbatimEnvironment{pcoden}{Verbatim}{commandchars=\\\{\},fontsize=\footnotesize,numbers=left,numbersep=0.5em,frame=lines,framesep=2ex,xleftmargin=1.5em}
%Above environments are to be used as normal, i.e. \begin{pcode}\end{pcode} or \begin{pcoden}\end{pcoden}

%paragraphed code from file
\CustomVerbatimCommand{\inputcode}{VerbatimInput}{fontsize=\footnotesize,frame=lines,framesep=2ex,xleftmargin=1.5em}
%paragraphed code from file with line numbering
\CustomVerbatimCommand{\inputcoden}{VerbatimInput}{fontsize=\footnotesize,numbers=left,numbersep=0.5em,frame=lines,framesep=2ex,xleftmargin=1.5em}
%Above commands are to be used as normal, i.e. \inputcode{dir/file.ext} or \inputcoden{dir/file.ext}


%%=============================
%% Counters
%%=============================
\numberwithin{equation}{section}
\numberwithin{figure}{section}
\numberwithin{exercise}{section}
\renewcommand\theenumi{\roman{enumi}}


%%=============================
%% Defined Math Operators
%%=============================

\DeclareMathOperator{\card}{card}
\DeclareMathOperator{\conv}{conv}
\DeclareMathOperator{\diam}{diam}
\DeclareMathOperator{\dist}{dist}
\DeclareMathOperator{\ddiv}{div}
\DeclareMathOperator{\esup}{ess \sup}
\DeclareMathOperator{\im}{Im}
\DeclareMathOperator{\interior}{int}
\DeclareMathOperator{\sgn}{sgn}
\DeclareMathOperator{\supp}{supp}
\DeclareMathOperator{\vol}{vol}
\DeclareMathOperator{\grad}{\nabla}


%%=============================
%% Defined Abreviations
%%=============================

\def\FFW{{\rm \kern-.08em{F}\kern-.15em\lower.3ex\hbox{\tiny{F}}\kern-.1em \lower-.1ex\hbox{\footnotesize{W}}} }

%%Spaces
\def\C{\ensuremath{\mathbb{C}}}
\def\Cinv{\ensuremath{\C^{-1}}}
\def\R{\ensuremath{\mathbb{R}}}


%%Colors
%\definecolor{string}{rgb}{0.7,0.0,0.0}

\newcommand{\T}{\mathcal{T}}
\newcommand{\E}{\mathcal{E}}
\newcommand{\K}{\mathcal{K}}
\newcommand{\M}{\mathcal{M}}
\newcommand{\norm}[2]{\lVert #1 \rVert_{#2}}



