\section{Graphical Output}
\label{sect:GraphicalOutput}

\noindent A very convenient way of getting graphical output of any kind from the structure \code{p} is to use the script \code{show.m} located at \path{.\evaluation\}. The function is called with a token, defining what is to be drawn, and the structure \code{p}, i.e., \code{p = show('token',p);}. Here \textit{token} is to be replaced by one of the following options:

\begin{longtable}{p{0.2\textwidth}p{0.7\textwidth}}
\textit{token} & Description\\\hline\\[-1ex]
\code{drawU} & Draw the solution $u$ on the underlying
               grid of each level of the refinement, saved in
               structure \code{p}.\\
\code{drawGradU} & Draw the gradient vector field $\grad u$ on the
                   underlying grid of each level of the refinement,
                   saved in structure \code{p}. \\
\code{drawError} & Draw the error development of the discrete
                   solution, as saved in structure \code{p}. Here one
                   has to specify which error is to be drawn by using
                   the parameters \verb"_estimatedError",
                   \verb"_L2error"
                   or \verb"_H1semiError" which are added to
                   \code{drawError}, i.e.,
                   \verb"p = show('drawError_L2error',p);".\\
\code{drawGrid} & Draw the grid of each level of the refinement,
                  saved in structure \code{p}.\\
\end{longtable}

\noindent Be aware that \code{drawU} and \code{drawGradU} depend on the PDE solver used, i.e., these functions can be found in the \code{script} folder of the \code{PDEsolver} used; whereas \code{drawGrid} and \code{drawError} are generic.\medskip\\
For all of the above tokens, the \code{show} function can display the data iteratively from the first level to the last or just show the last level of the refinement by setting \code{p.params.output.showIteratively} to \code{true} or \code{false}, default is \code{false}.\medskip\\
To save the figures, set \code{p.params.output.saveFigures} to \code{true}. The resulting files can
be found in a subfolder of \path{.\results\}, which describes the current problem. If you have choosen to show the data iteratively, a figure for each level is saved, except for the \code{drawError} token, where the error development from the first to the last level is saved in one figure.\\
The default file type is \code{fig}. See the following table for more parameters or MATLAB's help for the command \code{print}.\\

\begin{tabular}{p{0.2\textwidth}p{0.7\textwidth}}
filetype & description\\\hline\\[-1ex]
\code{fig}      & Saves a MATLAB figure.\\
\code{jpeg}     & Saves a JPEG file.\\
\code{epsc}     & Saves an encapsulated postscript color file.\\
\code{epsc2}    & Saves an encapsulated postscript level 2 color file.\\
\code{png}      & Saves a png file.\\\\[-1ex]
\end{tabular}

\noindent There are a number of predefined parameters to customize the graphical output for any of the above \textit{token}. Those parameters are also set by \code{defaultParametersOutput} during initialization of the \FFW.\newpage

\noindent Parameters for \code{drawU}:\\

\begin{tabular}{p{0.2\textwidth}p{0.1\textwidth}p{0.59\textwidth}}
parameter & type & description\\\hline\\[-1ex]
\code{drawInfo}    & boolean & Print labels describing the problem, e.g., caption and
                               degree of freedom. Default is \code{true}.\\
\code{drawWalls}   & boolean & Relevant only when using the PDE solver \code{P1P0} and
                               \code{RT0P0}. Default is \code{true}.\\
\code{myColor}     & char    & Color of the solution on the grid. Relevant only when
                               solving elasticity problems. Default is \code{'k'} for black.
                               For the color coding see MATLAB's help.\\
\code{lineWidth}   & integer & Line width of the solution. Relevant only when solving
                               elasticity problems. Default is \code{1}.\\
\code{factor}      & integer & Scales the solution by this factor. Relevant only when
                               solving elasticity problems. Default is \code{1000}.\\\\[-1ex]
\end{tabular}
\bigskip

\noindent Parameters for \code{drawGradU}:\\

\begin{tabular}{p{0.2\textwidth}p{0.1\textwidth}p{0.59\textwidth}}
parameter & type & description\\\hline\\[-1ex]
\code{drawInfo}    & boolean & Print labels describing the problem, e.g., caption and
                               degree of freedom. Default is \code{true}.\\
\code{localRes}    & integer & Determines the length of the gradients drawn.
                               Default is \code{10}.\\\\[-1ex]
\end{tabular}
\bigskip

\noindent Parameters for \code{drawError}:\\

\begin{longtable}{p{0.2\textwidth}p{0.1\textwidth}p{0.59\textwidth}}
parameter & type & description\\\hline\\[-1ex]
\code{drawInfo}    & boolean & Print labels describing the problem, e.g., caption and
                               degree of freedom. Default is \code{true}.\\
\code{myColor}     & char    & Color of the graph. Default is \code{'k'} for black.
                               For the color coding see MATLAB's help.\\
\code{lineStyle}   & char    & Determines the line style. Default is \code{'-'}.
                               For different styles see MATLAB's help.\\
\code{marker}      & char    & Determines the marker style. Default is \code{'x'}.
                               For different styles see MATLAB's help.\\
\code{minDoF}      & integer & Minimal number of degrees of freedom to start the plot with.
                               Default is \code{1}.\\                               
\code{plotGrid}    & boolean & Determines whether the axis grid is shown. Default is
                               \code{true}.\\
\code{holdIt}      & boolean & Determines whether the graph is to be appended to the current figure or
                               the figure is cleared before drawing. Default is
                               \code{true}.\\
\code{name}        & string  & Append the submitted name to the legend.
                               Default is \code{[]}.\\
\code{fontSize}    & integer & Size of all fonts in the figure. Default is the set fontsize of the
                               figure.\\
\code{setScales}   & boolean & Round the log-log plot to integer exponents.
                               Default is \code{false}.\\
\code{getConvRate} & boolean & Calculate the convergence rate and save it in the structure.
                               Default is \code{true}.\\
\code{drawConvRate}& boolean & Draw the convergence rate in the figure
                               (\code{getConvergenceRate} does not necessarily have to be \code{true}).
                               Default is \code{false}.\\
\code{degree}      & integer & Set the order of the integration routines of the error. Increased value
                               means more accuracy, decreased value means more performance.
                               Default is \code{19}.\\\\[-1ex]
\end{longtable}
\bigskip


\noindent Parameters for \code{drawGrid}:\\

\begin{tabular}{p{0.2\textwidth}p{0.1\textwidth}p{0.59\textwidth}}
parameter & type & description\\\hline\\[-1ex]
\code{drawInfo}    & boolean & Print labels describing the problem, e.g., caption and
                               degree of freedom. Default is \code{true}.\\
\code{color}       & char    & Color of the grid. Default is \code{'k'} for black.
                               For the color coding see MATLAB's help.\\
\code{lineWidth}   & integer & Determines the line width of the grid. Default is \code{1}.\\\\[-1ex]
\end{tabular}
