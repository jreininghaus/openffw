\subsection{Gauss Quadratur}
To integrate the right hand side of the partial differential equation
\begin{equation*}
\textrm{rhs} = \int_{\Omega} f \varphi\; dx + \int_{\Gamma_N} g \varphi\; dx
\end{equation*}
we need to use quadrature formulas. On the one hand we have to
calculate integrals in 1D for the neumann boundary and on the other hand
we have to solve integrals in 2D numerically.
In order to do this and for integration various types of integrals
we developed the following interface.

\subsubsection{Interface integrand}$ $\\
File: \path{.\algorithms\integrate\integrand.m}
\begin{pcode}
function val = integrate(parts, curLvl,  degree, integrand, p)
% input:  parts - nodes for elements or edges
%         curLvl - current level
%         degree - the integration will be exact for all polynomials 
%                  up to total degree 'degree'
%         p - FFW
% output: val - values of the integrand per element or edge [nrParts n m] 
\end{pcode}

\noindent
The integration interface can also handle functions $f:\mathbb{R}^k \mapsto\mathbb{R}^{m,n}$ in the sense that it integrates each component separately.\\
The function handle integrand has to be of the following form.
\begin{pcode}
function val = integrand(pts,curPart,curLvl,p)
\end{pcode}
Where the output has to be of the form \code{[n m length(x)]}.\\
For the integration of the right hand side there already exist two function handles.\\[1.5ex]
File: \path{.\algorithms\integrate\funcHandleRHSVolume.m}\\
File: \path{.\algorithms\integrate\funcHandleRHSNb.m}\\[1.5ex]
In both cases the function handle returns a matrix of the form\\ \code{[nrElements nrBasisFunctions]}
and the parameter \code{degree} is stored in
\begin{pcode}
p.params.rhsIntegtrateExactDegree
\end{pcode}
The \code{degree} parameter for the predefined function handles to calculate the energy and $L^2$ errors
can be modified in
\begin{pcode}
p.params.errorIntegtrateExactDegree
\end{pcode}
\bigskip

\noindent
The integration interface uses the following quadrature formulas.

\subsubsection{Gauss-Legendre formula}$ $\\
File: \path{.\algorithms\integrate\getGaussPoints.m}
\begin{pcode}
function [x,w] = getGaussPoints(n)
% input:   n - number of points
% output:  x - Gauss points [n 2]
%          w - Gauss weights [n 1]
\end{pcode}

\medskip
\noindent
We want to integrate numerically  a given function over the reference edge $\textrm{conv}\{0,1\}$
\begin{equation*}
 I(f) = \int_0^1\omega(x)f(x)\;dx
\end{equation*}
The integration formula has the form
\begin{equation*}
    \widetilde{I}(f) := \sum_{i=1}^n\omega_i f(x_i)\; ,
\end{equation*}
where $\omega_i$ are the weights and $x_i$ are the Gauss points.\\
The Gauss points are the roots of orthogonal polynomials. The weights are chosen such that the
formula is optimal for all polynomials up to total degree $p=2n-1$.\\
When we choose $\omega(x)=1$ and the interval $[-1,1]$ we get the Gauss-Legendre formula, where
the Gauss points are the roots of the n-th Legendre polynomial.\\
The following two theorems show how to calculate the Gauss points $x_i$ and weights $\omega_i$ efficiently for arbitrary $n\in\mathbb{N}$.
%
\begin{theorem}
The roots $x_i,\; i=1,\ldots,n$ of the n-th orthogonal polynomial $p_n$ are the
eigenvalues of the tridiagonal matrix
\begin{equation*}
J_n := \left[
\begin{array}{ccccc}
    \delta_1  & \gamma_2 &   &   & \\
    \gamma_2  & \delta_2 & . &   & \\
              & .        & . & . & \\
              &          & . & . &\gamma_n \\
              &          &   & \gamma_n & \delta_n \\
\end{array}
\right]\; .
\end{equation*}
Where the coefficients are recursively defined by $\delta_i,\gamma_i$
\begin{equation*}
\begin{split}
  \delta_{i+1} &:= (x p_i, p_i) / (p_i, p_i) \quad\textrm{ for}\; i\geq 0\\
  \gamma_{i+1}^2 &:= \left\{
\begin{array}{ll}
    1 & \textrm{for} \; i=0 \\
    (p_i,p_i)/(p_{i-1},p_{i-1}) & \textrm{for}\; i\geq 1\\
\end{array}\right.\; .
\end{split}
\end{equation*}
\end{theorem}
\noindent
For the Gauss-Legendre formula the coefficients are
\begin{equation*}
\delta_{i+1} = 0\; \textrm{ for}\; i\geq 0 \; \textrm{and} \; \gamma_{i+1} = \frac{i}{\sqrt{4i^2-1}}\;\textrm{ for}\; i\geq 1\; .
\end{equation*}
Therefore we can calculate the Gauss points $x_i$ with the following matlab lines.
\begin{pcode}
gamma = [1 : n-1] ./ sqrt(4*[1 : n-1].^2 - ones(1,n-1) );
[V,D] = eig( diag(gamma,1) + diag(gamma,-1) );
x = diag(D);
\end{pcode}
%
\begin{theorem}
It holds $ w_k = (v_1^{(k)})^2,\; k = 1,\ldots,n$, if $v^{(k)}=(v_1^{(k)},\ldots,v_n^{(k)})$ are the eigenvectors to the eigenvalue $x_k$ of $J_k$ with the norm factor $\lvert v^{(k)}\rvert = \int_{a}^b\omega(x)\; dx$.
\end{theorem}
For the Gauss-Legendre-formula there holds $\int_{a}^b\omega(x)\; dx = \int_{-1}^11\; dx=2$. E.g. we must multiply the weights with factor 2.
\begin{pcode}
w = 2*V(1,:).^2;
\end{pcode}
After that we have to transform the interval $[-1,1]$ to the reference edge $[0,1]$. Pay attention to the fact that there holds $\int_{0}^11\; dx=1$ and therefore we must multiply the weights with factor 1/2.
\begin{pcode}
x = .5 * x + .5;
w = .5 * w';
\end{pcode}

\subsubsection{Conical-Product formula}$ $\\
File: \path{.\algorithms\integrate\getConProdGaussPoints.m}
\begin{pcode}
function [x,w] = getConProdGaussPoints(n)
% input:   n - Number of Gauss points in 1D
% output:  x - Gauss points [n^2 2]
%          w - weights [n^2 1]
\end{pcode}

\medskip
\noindent
The Conical-Product formula is a quadrature formula especially for triangles. Here we use this formula for the reference triangle with the nodes $(0,0),(0,1)$ and $(1,1)$. The Conical-Product formula is a combination of 1D Gauss points.\\
The composite formula for the quadrangle is simply the cartesian product of the 1D Gauss-Legendre formula.
Therefor it is our aim to transform the quadrangle to an triangle by transforming the coordinates.
\begin{equation*}
\begin{split}
    x_1 &= y_1\\
    x_2 &= y_2(1-y_1) = y_2(1-x_1)\; .
\end{split}
\end{equation*}
From $0\leq x_1\leq 1$ and $0\leq x_1\leq 1-x_2$ we conclude $0\leq y_i\leq 1,\; i=1,2$. By this transformation we transformed the quadrangle to the triangle.\\
The Conical-Product formula is therefor a combination from the 1D Gauss formulas for the integrals
\begin{equation*}
\begin{split}
    \int_0^1 f(r)\; dr \approx \sum_{i=1}^n a_i f(r_i) \; , \\
    \int_0^1 (1-s)f(s)\; ds \approx \sum_{j=1}^n b_j f(s_j) \; .\\
\end{split}
\end{equation*}
The first integral can be calculated as above with the Gauss-Legendre formula. For the second integral we need a different Gauss formula. For the weight function $\omega(x) = (1-x)$ the Gauss-Jacobi formula is the right choice. The roots of the corresponding Jacobi polynomials can be calculated analogously to the roots of the Gauss-Legendre polynomials. The coefficients for the Jacobi polynomials are
\begin{equation*}
\delta_i = \frac{-1}{4i^2-1}\quad\textrm{and}\quad \gamma_{i+1} = \sqrt{\frac{i(i+1)}{2(i+1)-1}}\; .
\end{equation*}
%
\begin{pcode}
delta = -1./(4*(1 : n).^2-ones(1,n));
gamma = sqrt((2 : n).*(1 : n-1)) ./ (2*(2 : n)-ones(1,n-1));
[V,D] = eig( diag(delta)+diag(gamma,1)+diag(gamma,-1) );
s = diag(D);
b = 2*V(1,:).^2;
\end{pcode}
As above we have to transform the Gauss points to the interval $[0,1]$. Therefor we must also
multiply the weights of the Gauss-Jacobi formula with factor 1/4.
\begin{pcode}
r = .5 * r + .5;
a = .5 * a';
s = .5 * s + .5;
b = .25 * b';
\end{pcode}
The Conical-Product formula therefor has the $n^2$ Gauss points and weights
\begin{equation*}
    ( s_j , r_i(1-s_j) ) \quad a_i b_j \quad i=1,\ldots,n\quad j=1,\ldots,n\;.
\end{equation*}
\begin{pcode}
s = repmat(s',n,1); s = s(:);
r = repmat(r,n,1);
x = [ s , r.*(ones(n^2,1)-s) ];
w = a*b';
w = w(:);
\end{pcode}
