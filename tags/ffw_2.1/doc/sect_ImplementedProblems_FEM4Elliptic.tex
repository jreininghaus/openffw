\subsection{FEM for Elliptic PDEs}

The general form of an elliptic PDE is given through
\begin{align}\label{sect:FEMForEllipticProblems.eq.strongForm}
-\ddiv(\kappa\cdot\nabla u) + \lambda\cdot\nabla u + \mu u &= f &&\text{in } \Omega, \nonumber\\
u &= u_D &&\text{on } \Gamma_D, \\
\nabla u\cdot\nu &= g &&\text{on } \Gamma_N, \nonumber
\end{align}
with $u_D\in~H^1(\Omega;\R)$, $f\in~L^2(\Omega;\R)$, $g\in~L^2(\Gamma_N;\R)$, $\kappa\in~L^\infty(\Omega;\R^{2\times 2})$, $\lambda\in~L^{\infty}(\Omega,\R^2)$ and $\mu\in~L^\infty(\Omega,\R)$.

\medskip

\noindent The corresponding weak form of \eqref{sect:FEMForEllipticProblems.eq.strongForm} reads: Find $u\in V$ such that
\begin{align}
\label{sect:FEMForEllipticProblems.eq.weakForm}
	\int_\Omega \left( \nabla u\cdot(\kappa\cdot\nabla v) + \lambda\cdot\nabla u\,v + \mu u\,v\right )\,dx = \int_\Omega f\,v\,dx + \int_{\Gamma_N} g\,v\,ds_x.
\end{align}
for all $v\in V$.

\medskip

\noindent To get an approximation of the solution of \eqref{sect:FEMForEllipticProblems.eq.weakForm} several discretization methods are available. One can choose a conforming method where the discrete ansatz-space $V_h$ is a subset of $V$, i.e. $V_h\subset V$, and by that all functions $v_h\in V_h$ are globally continuous. Here one has the choice between the piecewise linear ($P_1$), quadratic ($P_2$) or cubic ($P_3$) FE-ansatz-spaces. Another method which can be taken is the non-conforming Crouzeix-Raviart ($CR$). Here the piecewise linear ansatz-space $V_h$ is not a subset of $V$. Continuity is only enforced on the midpoints of the edges of the underlying triangulation. To get a full detailed description of the theoretical and technical aspects of these methods one may have a look in \cite{Bra}. Here, these standard methods are well introduced. 

\bigskip

\noindent If the main interest is on an accurate stress or flux approximation and some strict equilibration condition rather than the displacement, it might be advantageous to consider an operator split: Instead of one partial differential equation of order $2m$ one considers two equations of order $m$. To be more precise, given one equation in an abstract form $\mathcal{L}u = G$ with some differential operator $\mathcal{L} = \mathcal{A}\mathcal{B}$ composed of $\mathcal{A}$ and $\mathcal{B}$, define $p := \mathcal{B}u$ and solve the two equations $\mathcal{A}p = G$ and $\mathcal{B}u = p$.

\smallskip

\noindent With this splitting, the mixed formulation of \eqref{sect:FEMForEllipticProblems.eq.strongForm} reads: Find $(\sigma,u)\in\Sigma\times V$ such that
\begin{align}\label{sect:FEMForEllipticProblems.eq.mixedStrongForm}
  -\ddiv\sigma + \lambda\cdot(\kappa^{-1}\cdot\sigma) + \mu u &= f &&\text{in } \Omega, \nonumber\\
  \sigma &= \kappa\cdot\nabla u &&\text{in } \Omega,\\
  u &= u_D &&\text{on } \Gamma_D, \nonumber\\
  (\kappa^{-1}\cdot\sigma)\cdot\nu &= g &&\text{on } \Gamma_N\nonumber
\end{align}
with the corresponding weak form
\begin{align}\label{sect:FEMForEllipticProblems.eq.mixedWeakForm}
  \int_\Omega \left( -\ddiv\sigma\,v + \lambda\cdot(\kappa^{-1}\cdot\sigma)\,v + \mu u\,v\right)\,dx &= \int_\Omega f\,v\,dx, \\
  \int_\Omega \left((\kappa^{-1}\cdot\sigma)\cdot\tau + \ddiv\tau\, u\right)\,dx &= \int_{\Gamma_D} u_D(\tau\cdot\nu)\,ds_x,\nonumber
\end{align}
for all $v\in V$ and all $\tau\in\Sigma$.

\bigskip

\noindent In the \FFW the lowest order Raviart-Thomas mixed finite element method ($RT_0-P_0$) of \eqref{sect:FEMForEllipticProblems.eq.mixedWeakForm} is implemented. Here the discrete flux $\sigma_h\in RT_0(\T)\subset H(\ddiv;\Omega)$ is a linear vector field of the form $a\cdot x+b$ with $a\in\R^2$ and $b\in\R$ which is continuous in normal direction. The discrete displacement $u_h\in P_0(\T)\subset L^2(\Omega)$ is piecewise constant. 

\medskip

\noindent An theoretical overview over mixed finite element methods can be found in \cite{Ar,Bra}. The work \cite{BahCC} gives a detailed description of the implemented $RT_0-P_0$ method with focus on the numerical realization.