\section{Data Structures}
\label{sect:DataStructures}

\noindent All the functions that calculate the grid information needed in the \FFW, e.g. area of elements, length of edges, outer unit normals of the boundary, and the local gradients for $P_1$ and $P_1^{NC}$ basis functions are located at \path{.\algorithms\enum}.\medskip

\noindent The information returned by those functions, the enumerated data, is stored in matrices. In these matrices a row or column number corresponds to the number of an element, node, edge etc. If there is no information on an element, node, edge etc. the corresponding entry is zero. For example a boundary edge will have only one none zero entry in \code{e4ed}, since there is only one element which contains it.\medskip

\noindent In the names of the data matrices \code{e} stands for \textbf{e}lements, \code{n} for \textbf{n}odes, \code{ed} for \textbf{ed}ges, \code{Db} for \textbf{D}irichlet\textbf{b}oundary, \code{Nb} for \textbf{N}eumann\textbf{b}oundary and \code{4} means '\textbf{for}'. For example, the matrix \code{e4ed} contains elements for an edge. In the following these names are used as well as the abbreviations \code{nrElems},
\code{nrNodes} etc., standing for number of elements, number of nodes etc.\,.\medskip

\noindent In a triangulation in the \FFW the geometric primitives (elements, nodes and edges) are numbered uniquely. Dirichlet and Neummann edges are also numbered in this way. The numbering of elements, nodes, Dirichlet and Neumann edges is defined by \code{n4e}, \code{c4n}, \code{Db} and \code{Nb}, respectively, whereas the edge numbers are created in \path{.\enum\getEd4n}. All the other information (e.g. Normals, Tangents) is not numbered and is used as attributes of the information above.\medskip

\noindent For the geometric primitive one has not only a global number, but most times also a local one. For example the global number of a node is defined by the row number in \code{c4n}, and for each element its nodes are locally numbered from one to three by there order in the corresponding row in \code{n4e}.\medskip

\noindent To calculate all the grid information the only initial data needed is \code{n4e}, \code{c4n}, \code{Db} and \code{Nb}. Please note that not all information is calculated using directly the initial data. For example \code{e4n} needs only the information in \code{n4e}, but \code{ed4n} is created using the information in \code{e4n}.\bigskip\\


\subsection{Initial Data}$ $\\

\begin{longtable}{p{0.06\textwidth}p{0.3\textwidth}p{0.54\textwidth}}
\code{Name} &Dimension  &Description\\ \hline
\code{c4n}  &\code{[nrNodes 2]} 
& Each row defines a node at the coordinates given by the values. Where the first column contains the x-coordinate and the second the y-coordinate. The number of the defined node is the number of the row.\\
\code{n4e}  &\code{[nrElems 3]}
& Each row defines an element with the nodes corresponding to the entries as vertices. The vertices have to be entered counter clockwise. The number
of the defined element is the number of the row.\\
\code{Db}   &\code{[nrDirichletEdges 2]}
& Each row defines the edge between the two nodes corresponding to the entries as a Dirichlet edge. In each row the nodes have to be in the same order as in the corresponding element.\\
\code{Nb}   &\code{[nrNeumannEdges 2]}
& Each row defines the edge between the two nodes corresponding to the entries as a Neumann edge. In each row the nodes have to be in the same order as in the corresponding element.
\end{longtable}
\bigskip

\subsection{Enumerated Data}$ $\\

\begin{longtable}{p{0.2\textwidth}p{0.25\textwidth}p{0.45\textwidth}}
Name&Dimension&Description\\ \hline
\code{e4n}  &\code{[nrNodes nrNodes]}
&Each entry $(j,k)$ contains the number of the element which has the nodes $j$ and $k$ counter clockwise as vertices.\\
\code{ed4n} &\code{[nrNodes nrNodes]}
&Each entry $(j,k)$ contains the number of the edge between the nodes $j$ and $k$.\\
\code{ed4e} &\code{[nrElems 3]}
&Each row contains the edge numbers of the corresponding element.\\
\code{n4ed} &\code{[nrEdges 2]}
&Each row contains the node numbers of the corresponding edge.\\
\code{e4ed} &\code{[nrEdges 2]}
&Each row contains the element numbers of the elements sharing the corresponding edge in descending order.\\
\code{DbEdges}      &\code{[nrDbEdges 1]}
&Each row contains the number of a Dirichlet edge corresponding to the row in \code{Db}.\\
\code{NbEdges}      &\code{[nrNbEdges 1]}
&Each row contains the number of a Neumann edge corresponding to the row in \code{Nb}.\\
\code{area4e}       &\code{[nrElems 1]}
&Each row contains the area of the corresponding element.\\
\code{midpoint4e}   &\code{[nrElems 2]}
&Each row contains the coordinates of the midpoint of the corresponding element.\\
\code{midpoint4ed}  &\code{[nrEdges 2]}
&Each row contains the coordinates of the midpoint of the corresponding edge.\\
\code{tangents4e}   &\code{[3 2 nrElems]}
&Each $3\times2$ matrix contains the coordinates of the three unit tangents of the corresponding element.\\
\code{normals4e}    &\code{[3 2 nrElems]}
&Each $3\times2$ matrix contains the coordinates of the three outer unit normals of the corresponding element.\\
\code{normals4DbEd} &\code{[nrDbEdges 2]}
&Each row contains the coordinates of the outer unit normals of the corresponding Dirichlet edge.\\
\code{normals4NbEd} &\code{[nrNbEdges 2]}
&Each row contains the coordinates of the outer unit normals of the corresponding Neumann edge.\\
\code{length4ed}    &\code{[nrEdges 1]}
&Each row contains the length of the corresponding edge.\\
\code{area4n}       &\code{[nrNodes 1]}
&Each row contains the area of the node patch of the corresponding node.\\
\code{angles4e}     &\code{[nrElems 3]}
&Each row contains the inner angles at each node of the corresponding element in the order given in the rows of \code{n4e}.\\
\code{angle4n}      &\code{[nrNodes 1]}
&Each row contains the sum of all inner angles at the corresponding node.\\
\code{grad4e}       &\code{[3 2 nrElems]}
&Each $3\times2$ matrix contains the three local gradients of the three $P_1$ basis functions not completely zero on the corresponding element.\\
\code{gradNC4e}     &\code{[3 2 nrElems]}
&Each $3\times2$ matrix contains the three local gradients of the three $CR$ basis functions not completely zero on the corresponding element.
\end{longtable}
\bigskip

\noindent The following figure shows how one can get from one geometric primitive to another one. To get for example all edges for one element just look at the corresponding row in \code{ed4e}.

\begin{figure}[ht!]
\vspace{5ex}
\begin{center}
\setlength{\unitlength}{3.5cm}
\begin{picture}(1.5,0.9)
\put(0,0){\line(1,0){1.2}}
\put(0,0){\line(2,3){0.6}}
\put(1.2,0){\line(-2,3){0.6}}
\put(-0.3,-0.1){\textbf{N}odes}
\put(1.25,-0.1){\textbf{E}dges}
\put(0.4,0.95){\textbf{E}lements}
\put(0.105,0.2){\vector(2,3){0.4}}
\put(0.1,0.5){e4n}
\put(0.5,0.7){\vector(-2,-3){0.4}}
\put(0.32,0.35){n4e}
\put(0.735,0.74){\vector(2,-3){0.4}}
\put(1,0.4){ed4e}
\put(1.05,0.18){\vector(-2,3){0.4}}
\put(0.6,0.46){e4ed}
\put(0.16,0.03){\vector(1,0){0.9}}
\put(0.6,0.05){ed4n}
\put(1,-0.03){\vector(-1,0){0.9}}
\put(0.4,-0.11){n4ed}
\end{picture}
\end{center}
$ $\\
\caption{This diagram illustrates the relations, realized by the enumerated data above, between elements abbreviated by \code{e}, edges abbreviated by \code{ed} and nodes abbreviated by \code{n}.}\label{sect:DataStructures.fig.DataRelations}
\end{figure}


\noindent Figure~\ref{sect:DataStructures.fig.ExampleEnumeration} shows a triangulation of two triangles generated by the \FFW\!. The numbering is used to illustrate the structure of the data created in \path{.\algorithms\enum} by means of some examples below.

\begin{figure}[h!]
\vspace{2ex}
\begin{center}
\setlength{\unitlength}{3.5cm}
\begin{picture}(2.2,0.9)
\put(0,0){\line(1,0){1.4}}
\put(0,0){\line(3,4){0.7}}
\put(1.4,0){\line(3,4){0.7}}
\put(0.7,0.93){\line(1,0){ 1.4}}
\put(0.7,0.93){\line(3,-4){0.7}}
\put(0.65,0.3){I}
\put(0.58,0.325){\line(2,3){.08}}
\put(0.58,0.325){\line(2,-3){.08}}
\put(0.74,0.325){\line(-2,-3){.08}}
\put(0.74,0.325){\line(-2,3){.08}}
\put(1.35,0.57){II}
\put(1.30,0.595){\line(2,3){.08}}
\put(1.30,0.595){\line(2,-3){.08}}
\put(1.46,0.595){\line(-2,-3){.08}}
\put(1.46,0.595){\line(-2,3){.08}}
\put(-0.05,-0.1){1}
\put(-0.026,-0.075){\circle{0.1}}
\put(1.4,-0.1){2}
\put(1.424,-0.075){\circle{0.1}}
\put(0.65,0.96){4}
\put(0.676,0.985){\circle{0.1}}
\put(2.1,0.96){3}
\put(2.125,0.985){\circle{0.1}}
\put(0.7,-0.08){\it 1}
\put(0.66,-0.09){\line(1,0){0.13}}
\put(0.66,-0.09){\line(0,1){0.09}}
\put(0.79,-0.09){\line(0,1){0.09}}
\put(1.74,0.39){\it 2}
\put(1.78,0.32){\line(3,4){0.09}}
\put(1.78,0.32){\line(-4,3){0.09}}
\put(1.87,0.44){\line(-4,3){0.09}}
\put(0.24,0.41){\it 3}
\put(0.17,0.41){\line(3,4){0.09}}
\put(0.17,0.41){\line(4,-3){0.09}}
\put(0.26,0.53){\line(4,-3){0.09}}
\put(1.075,0.47){\it 4}
\put(1.10,0.58){\line(3,-4){0.09}}
\put(1.10,0.58){\line(-4,-3){0.09}}
\put(1.19,0.46){\line(-4,-3){0.09}}
\put(1.4,0.94){\it 5}
\put(1.37,1.02){\line(1,0){0.13}}
\put(1.37,1.02){\line(0,-1){0.09}}
\put(1.50,1.02){\line(0,-1){0.09}}
\end{picture}
\vspace{2ex}
\caption{Enumeration of nodes, edges and elements. The roman numbers in the rhombuses are the element numbers, the numbers in circles are the node numbers and the italic numbers in the rectangles are the edge numbers.}
\label{sect:DataStructures.fig.ExampleEnumeration}
\end{center}
\end{figure}


\noindent The following \code{n4e} creates a triangulation as shown in the Figure~\ref{sect:DataStructures.fig.ExampleEnumeration}.
\begin{pcode}
>> n4e
n4e =
     1     2     4
     2     3     4
\end{pcode}
Note that the order of the nodes is important in \code{e4n} but it is not in \code{ed4n}.
\begin{pcode}
>> e4n(2,4)
ans =
     1
>> e4n(4,2)
ans =
     2
>> ed4n(2,4)
ans =
     4
>> ed4n(4,2)
ans =
     4
\end{pcode}
One can also get the information for more than one item at a time.
\begin{pcode}
>> n4ed([3 4],2)
ans =
     4
     4
\end{pcode}
If there is only one element containing the edge, i.e., the edge is a boundary edge, then the second entry is zero.
\begin{pcode}
>> e4ed([1 4],:)
ans =
     1     0
     2     1
\end{pcode}
The edges are ordered counter clockwise for each element. The first one in each element is the one
between the first end the second node (\emph{not} opposite to the first node).
\begin{pcode}
>> ed4e(1,:)
ans =
     1     4     3
\end{pcode}
\bigskip

\subsection{Overview of the Enumeration Functions}$ $\\

\noindent\emph{getE4n.m}\smallskip\\
The function \code{getE4n} returns a \code{[nrNodes nrNodes]} sparse matrix. \emph{Note}: The sparsity constant is bounded due to the used mesh generation. The input is \code{n4e}. In this matrix each entry $(j,k)$ is the number of the element, whose boundary contains the nodes $j$ and $k$ in counter clockwise order as vertices or zero if the there is no such element. Since the nodes in \code{n4e} are oriented counter clockwise \code{e4n} gives you the number of the row in which the sequence $j\;k$ is found. Note that in this context $k\;i\;j$ also contains the sequence $j\;k$. To find the patch of a node $k$, i.e, all elements containing node $k$, just get the non zero entries of the $k$-th row or column.\bigskip


\noindent\emph{getEd4n.m}\smallskip\\
The function \code{getEd4n} returns a symmetric \code{[nrNodes nrNodes]} sparse matrix. \emph{Note}: The sparse constant is bounded due to the used mesh generation. The input is \code{e4n} generated by the function \code{getE4n}. The output matrix \code{ed4n} contains the numbers of the edges between two nodes or zero if the two nodes are not on one edge. In the sense that for node $j$ and node $k$ the entry $(j,k)$, and $(k,j)$ respectively, is the corresponding edge number. The numbering of the edges is arbitrarily generated in this function. The input is \code{n4e}.\bigskip


\noindent\emph{getN4ed.m}\smallskip\\
The function \code{getN4ed} returns a \code{[nrEdges 2]} matrix. the input is \code{ed4n} generated by the function \code{getEd4n}. The output matrix contains in each row the number of the two nodes that are the endpoints of the edge corresponding to the row number.\bigskip


\noindent\emph{getEd4e.m}\smallskip\\
The function \code{getEd4e} returns a \code{[nrElems 3]} matrix. The input is \code{n4e} and \code{ed4n} produced by the function \code{getEd4n}. This matrix contains in each row $j$ the number of the three edges of element $j$. The edge numbers are the ones generated in the function \code{getEd4n}. The edges are ordered counter clockwise beginning with the edge between the first and the second node in \code{n4e}.\bigskip


\noindent\emph{getE4ed.m}\smallskip\\
The function \code{getE4ed} returns a \code{[nrEdges 2]} matrix. The input is \code{e4n} and \code{n4ed} produced by the functions \code{getE4n} and \code{getN4ed}, respectively. The matrix contains in each row $j$ the element numbers of the elements which share the edge. If the edge is a boundary edge the second entry is zero.\bigskip


\noindent\emph{getArea4e.m}\smallskip\\
The function \code{getArea4e} returns a \code{[nrElems 1]} matrix. The input is \code{n4e} and \code{c4n}. The matrix contains in each row $j$ the area of the element corresponding to the element number $j$.\bigskip


\noindent\emph{getArea4n.m}\smallskip\\
The function \code{getArea4n} returns a \code{[nrNodes 1]} matrix. The input is \code{e4n} and \code{area4e} produced by the functions \code{getE4n} and \code{getArea4e}, respectively. The matrix contains in each row $j$ the area of the patch of node $j$.\bigskip


\noindent\emph{getLength4ed.m}\smallskip\\
The function \code{getLength4ed} returns a \code{[nrEdges 1]} matrix. The input is \code{c4n} and \code{n4ed} produced by the function \code{getN4ed}. The matrix contains in each row $j$ the length of the $j$-th edge, according to the edge numbers created in \code{getEd4n}.\bigskip


\noindent\emph{getDbEdges.m}\smallskip\\
The function \code{getDbEdges} returns a \code{[nrDbEdges 1]} matrix. The input is \code{Db} and \code{ed4n} produced by the function \code{getEd4n}. The Matrix contains the edge numbers of the edges belonging to the Dirichlet boundary.\bigskip


\noindent\emph{getNbEdges.m}\smallskip\\
The function \code{getNbEdges} returns a \code{[nrNbEdges 1]} matrix. The input is \code{Nb} and \code{ed4n} produced by the function \code{getEd4n}. The Matrix contains the edge numbers of the edges belonging to the Neumann boundary.\bigskip


\noindent\emph{getNormals4DbEd.m}\smallskip\\
The function \code{getNormals4DbEd} returns a \code{[nrDbEdges 2]} matrix. The input is \code{c4n} and \code{Db}. The matrix contains in each row $j$ the two coordinates of the outer unit normal at the $j$-th Dirichlet edge, corresponding to the order in \code{DbEdges}.\bigskip


\noindent\emph{getNormals4NbEd.m}\smallskip\\
The function \code{getNormals4NbEd} returns a \code{[nrNbEdges 2]} matrix. The input is \code{c4n} and \code{Nb}. The matrix contains in each row $j$ the two coordinates of the outer unit normal at the $j$-th Neumann edge, corresponding to the order in \code{NbEdges}.\bigskip


\noindent\emph{getNormals4e.m}\smallskip\\
The function \code{getNormnals4e} returns a \code{[3 2 nrElems]} matrix. The input is \code{c4n}, \code{n4e} and \code{length4ed} and \code{ed4e} produced by the functions \code{getLength4ed} and \code{getEd4e}. The three dimensional matrix contains in the $j$-th $3 \times 2$ matrix the coordinates of the unit outer normals for the three edges of element $j$. The order of the three rows corresponds two the order in \code{ed4e}.\bigskip


\noindent\emph{getTangents4e.m}\smallskip\\
The function \code{getNormnals4e} returns a \code{[3 2 nrElems]} matrix. the input is \code{c4n}, \code{n4e} and \code{length4ed} and \code{ed4e} produced by the functions \code{getLength4ed} and \code{getEd4e}. The three dimensional matrix contains in the $j$-th $3 \times 2$ matrix the coordinates of the unit tangents in counter clockwise direction for the three edges of element $j$, i.e., the direction of the edges.
The order of the three rows corresponds two the order in \code{ed4e}.\bigskip


\noindent\emph{getMidpoint4e.m}\smallskip\\
The function \code{getMidpoint4e} returns a \code{[nrElems 2]} matrix. The input is \code{c4n} and \code{n4e}. The matrix contains in each row $j$ the coordinates of the midpoint of element $j$.\bigskip


\noindent\emph{getMidpoint4ed.m}\smallskip\\
The function \code{getMidpoint4ed} returns a \code{[nrEdges 2]} matrix. The input is \code{c4n} and \code{n4ed}. The matrix contains in each row $j$ the coordinates of the midpoint of the $j$-th edge.\bigskip


\noindent\emph{getAngles4e.m}\smallskip\\
The function \code{getAngles4e} returns a \code{[3 nrElems]} matrix. The input is \code{tangents4e} produced by the function \code{getTangente4e}. The matrix contains in each column $j$ the three interior angles of element $j$. The order of the angles correspond to the order of the nodes in \code{n4e}, i.e. the angles at the first node of each element are in the first row, the one at the second in the second etc.\bigskip


\noindent\emph{getAngle4n.m}\smallskip\\
The function \code{getAngle4n} returns a \code{[nrNodes 1]} matrix. The input is \code{angles4e}, \code{n4e}, \code{nrElems} and \code{nrNodes}
where \code{angles4e} is generated by \code{getAngles4e}. The matrix contains in each row $j$ the angle at the node $j$, where the angle is $2\pi$ for inner nodes, and the angle inside the domain and between the two boundary edges containing the node for nodes at the boundary.\bigskip


\noindent\emph{getGrad4e.m}\smallskip\\
The function \code{getGrad4e} returns a \code{[3 2 nrElems]} matrix. The input is \code{c4n}, \code{n4e} and \code{area4e} produced by the function \code{getArea4e}. The three dimensional matrix contains in the $j$-th $3 \times 2$ matrix the coordinates of the local gradients of the three $P_1$ basis function which are not constantly zero on the $j$-th element. The first row in each $3 \times 2$ matrix contains the local gradient of the nodal basis function for the first node of the element according to the order in \code{n4e}, the second row the one of the nodal basis function for the second node etc.\,.\bigskip


\noindent\emph{getGradNC4e.m}\smallskip\\
The function \code{getGrad4e} returns a \code{[3 2 nrElems]} matrix. The input is \code{c4n}, \code{n4e} and \code{area4e} produced by the function \code{getArea4e}. The three dimensional matrix contains in the $j$-th $3 \times 2$ matrix the coordinates of the local gradients of the three $P_1^{NC}$ basis function which are not constantly zero on the $j$-th element. The first row in each $3 \times 2$ matrix contains the local gradient of the basis function corresponding to the first edge in \code{ed4e}, the second row the one of the second edge etc.\,.