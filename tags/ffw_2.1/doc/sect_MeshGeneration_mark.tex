\subsection{Mark}
In order to refine the mesh adaptively we first have to say where.
In difference to other implementations we generally mark only edges 
instead of elements. If we want to refine an element we therefor
have to mark all its edges.
An edge is refined depending on the local error indicators.
There are three different contributions to the total estimated error.
There is the local error on an edge $\eta_E$ (\code{etaEd}), on an element $\eta_T$ (\code{etaT}) and the 
local oscillations $\eta_{osc}$ (\code{etaOsc}). The oscillations on elements are refined \emph{bisec5} instead of
\emph{red}. To compute the set of marked edges $\M_\ell$ there are four different
strategies implemented.
%
%
\subsubsection{uniform} $ $\\
File: \path{.\algorithms\mark\uniform.m}\\[1.5ex]
All elements are refined uniformly \emph{red}. Therefore 
$\M_\ell = \mathcal{E}_\ell$ and no elements are marked for \emph{bisec5} refinement.
\begin{pcode}
refineEdges = true(nrEdges,1);
refineElemsBisec5 = false(nrElems,1);
\end{pcode}
%
\subsubsection{maximum} $ $\\
File: \path{.\algorithms\mark\maximum.m}\\[1.5ex]
The maximum algorithm defines the set $\M_\ell\subseteq\E_\ell$ of marked edges such that for all $E\in\M_\ell$
\begin{equation*}
\eta_{E} >  \theta \cdot \max_{K\in\E_\ell}\eta_{K},
\end{equation*}
where $\theta \in [0,1]$ is a constant (default: $\theta = 0.5$).
The MATLAB code is printed in the next line.
\begin{pcode}
refineEdges = (etaEd > thetaEd * max(etaEd));
\end{pcode}
\code{True} means that the edge is marked and \code{false} it
is not.
If elements should be refined all the edges of an element $T$ belong to the set $\M_\ell$ if
\begin{equation*}
\eta_{T} >  \theta \cdot \max_{K\in\T_\ell}\eta_{K}.
\end{equation*}
The code realizes this by first marking the elements and than marking all the edges of marked elements.
\begin{pcode}
I = (etaT > thetaT * max(etaT))';
refineElems(I) = true;
refineEdges4e = ed4e(refineElems,:);
refineEdges(refineEdges4e(:)) = true;
\end{pcode}
Elements that have large osscilations
(\code{etaOsc}) are treated analogously. The reason why they are refined with \emph{bisec5}
instead of \emph{red} is that \emph{bisec5} generates
a new node in the interior of the element.\bigskip

\noindent The maximum algorithm can be very inefficient. Suppose we have the
same error on each edge except of very few edges and on those edges the error
is very large. Then the maximum algorithm might only refine these few edges 
(cf. figure~\ref{sect:MeshGeneration.Mark.maximum.fig}).

\begin{figure}
\setlength{\unitlength}{6cm}
\begin{picture}(1, 1)
\qbezier(0.1, 0.2)(0.15,0.5)(0.2,0.2)
\qbezier(0.2, 0.2)(0.25,0.25)(0.3,0.2)
\qbezier(0.3, 0.2)(0.35,0.4)(0.4,0.2)
\qbezier(0.4,0.2)(0.45,0.5)(0.5, 0.3)
\qbezier(0.5,0.3)(0.525,0.6)(0.55, 1)
\qbezier(0.55, 1)(0.575,0.6)(0.6, 0.2)
\qbezier(0.6,0.2)(0.65,0.5)(0.7, 0.2)
\qbezier(0.7, 0.2)(0.75,0.4)(0.8,0.2)
\qbezier(0.8, 0.2)(0.85,0.25)(0.9,0.2)
\put(0.05,0.5){\line(1,0){0.9}}
\put(1,0.5){\tiny 0.5 max}
\put(0,0){\vector(0,1){1}}
\put(0,0){\vector(1,0){1}}
\put(1.05,0){\tiny edge or element}
\put(0.05,0.9){\tiny estimated error}
\end{picture}
\caption{The maximum algorithm might mark only a few edges.}
\label{sect:MeshGeneration.Mark.maximum.fig}
\end{figure}

\subsubsection{bulk}
The bulk algorithm defines the set $\M_\ell$ of marked edges such that
\begin{equation*}
 \sum_{E\in \M_\ell}\eta_E^2 \geq  \theta \cdot \sum_{K\in\E_\ell} \eta_K^2,
\end{equation*}
or it contains all the edges of elements $T\in \mathcal{K}_\ell$ that satisfy
\begin{equation*}
 \sum_{T\in\mathcal{K}_\ell}\eta_T^2 \geq  \theta \cdot \sum_{K\in\T_\ell} \eta_K^2,
\end{equation*}
It is important to see that the set  $\M_\ell$ of edges that
are selected by this condition is not unique.
Here we use an greedy approach. We iteratively take
those edges with the largest error. Using this approach 
guarantees that we have the smallest possible set of
marked edges that satisfy the bulk condition.
In the following the corresponding MATLAB lines are printed.
\begin{pcode}
[sortedEtaEd,I] = sort(etaEd,'descend');
sumEtaEd = cumsum(sortedEtaEd.^2);
k = find(sumEtaEd >= thetaEd * norm(etaEd,2)^2,1,'first');
[sortedEtaT,I] = sort(etaT,'descend');
sumEtaT = cumsum(sortedEtaT.^2);
k = find(sumEtaT >= thetaT * norm(etaT,2)^2,1,'first');
refineElems(I(1:k)) = true;
refineEdges4e = ed4e((refineElems | refineElemsBisec5),:);
refineEdges(refineEdges4e(:)) = true;
\end{pcode}
The problem that only a few edges or element are marked 
like with the maximum algorithm cannot happen here, see Figure~\ref{sect:MeshGeneration.Mark.bulk.fig}.

\begin{figure}
\setlength{\unitlength}{6cm}
\begin{picture}(1, 1)
\qbezier(0.1, 0.2)(0.15,0.5)(0.2,0.2)
\qbezier(0.2, 0.2)(0.25,0.25)(0.3,0.2)
\qbezier(0.3, 0.2)(0.35,0.4)(0.4,0.2)
\qbezier(0.4,0.2)(0.45,0.5)(0.5, 0.3)
\qbezier(0.5,0.3)(0.525,0.6)(0.55, 1)
\qbezier(0.55, 1)(0.575,0.6)(0.6, 0.2)
\qbezier(0.6,0.2)(0.65,0.5)(0.7, 0.2)
\qbezier(0.7, 0.2)(0.75,0.4)(0.8,0.2)
\qbezier(0.8, 0.2)(0.85,0.25)(0.9,0.2)
\put(0.05,0.25){\line(1,0){0.9}}
\put(1,0.25){\tiny bulk}
\put(0,0){\vector(0,1){1}}
\put(0,0){\vector(1,0){1}}
\put(1.05,0){\tiny edge or element}
\put(0.05,0.9){\tiny estimated error}
\end{picture}
\caption{With the bulk algorithm it cannot happen that only
         a few edges are marked.}
\label{sect:MeshGeneration.Mark.bulk.fig}
\end{figure}

\subsubsection{graded}
\textbf{file:} \path{.\algorithms\mark\graded.m}\\[1.5ex]
Refinement with graded grids is a a priori mesh refinement toward singular corner points.
In the past the so called $\beta$-graded grids were very popular. The a priori analysis consists of the following theorem.
%
\begin{theorem}
Let $\mathcal{T}$ be a regular triangulation of $\Omega=T_{ref}$ such that for given $N\in\mathbb{N}$ and $\beta>0$
\begin{enumerate}
\item $\mathcal{T}$ contains the element $T_0=conv\{(0,0),(N^{-\beta},0),(0,N^{-\beta})\}$.
\item For each $T\in \mathcal{T}\backslash\{T_0\}$ and all $x\in T$ one has $diam(T)\leq c\frac{1}{N}|x|^{1-\beta}$
\end{enumerate}
Then, if $\alpha + \beta > 2$, it follows that
\begin{equation*}
|| \nabla(u_{\alpha} - I_{\mathcal{T}}u_{\alpha})||_{L^2(\Omega)}\leq c N^{-min\{1,\alpha\beta\}}\; .
\end{equation*}
where $u_{\alpha}$ is the corner singularity function and $\alpha$ depends on the opening angle in the corner singularity.
\end{theorem}
\noindent
It is known that $\beta$-graded grids satisfy this conditions. But here we follow another approach. We simply refine our mesh with \emph{red}-\emph{green}-\emph{blue} refinement until these conditions are satisfied. This leads to simpler algorithms since $\beta$-graded grids are difficult to implement. Another advantage is that the angles of the triangulation only depend on the initial triangulation. The parameter $\beta$ can be modified by setting the value
\begin{pcode}
p.params.modules.mark.graded.beta
\end{pcode}
For the L-shaped domain the default value $1/3$ is optimal. It is important to say that the implemented graded algorithm
assumes that the problem has exactly one singularity located at the origin.
