\subsection{Initialization of the FFW}
\label{sect:QuickStart:Initialization}

\noindent
The initialization of the \FFW\! is called as follows:
\begin{pcode}
p = initFFW(pdeSolver,problem,mark,maxNrDoF,problemType)
\end{pcode}
or with optional Parameter
\begin{pcode}
p = initFFW(pdeSolver,problem,mark,maxNrDoF,problemType,solver,refine,estimate)
\end{pcode}

\noindent
After the initialization of the structure, in the following refered to as \code{p}, it is still possible to 
change specific parameters in the structure if needed.
In the initialization progress at first all parameters given with the function call as well as the default parameters
are stored in \code{p.params.}, then the function handles to the files needed in the AFEM loop are
stored in \code{p.statics.} and the problem definition is loaded. The geometry is stored in \code{p.level(end).geom}, see section~\ref{sect:ImplementedProblems_Geometries}. After that the problem type specific initialisation in \path{.\PDESolver\<problemType>\<problemType>_init.m} as well as the pde solver specific initialization routine in 
\path{.\PDESolver\<problemType>\<pdeSolver>\<pdeSolver_init.m>} are executed. The following table gives an overview 
of the possible parameters for \code{initFFW.m}.\bigskip

\begin{longtable}[h]{lll}
token & parameter & description \\\hline\\[-1ex]

\verb+pdeSolver+
&\verb+'P1'+	&         P1-Elliptic\\
&\verb+'P2'+	&         P2-Elliptic\\
&\verb+'P3'+	&         P3-Elliptic\\
&\verb+'CR'+	&         CR-Elliptic\\
&\verb+'P1P0'+	&       P0-P1-mixed-Elliptic (unstable element,\\
    & & works only with uniform refinement)\\
&\verb+'RT0P0'+	&      RT0-P0-mixed-Elliptic\\
&		&choose different Finite Element Methods\\
\\
\verb+problem+& <problem name>
		&a string which represents \\ %\verb+'Elliptic_Lshape_exact'+
 &		&the name of a file in the folder\\
&		&\verb+\problems\elliptic+\\
&		&or \verb+\problems\elasticity+\\
\\
\verb+mark+&
 \verb+'bulk'+	&a string which represents \\
&\verb+'graded'+& a marking algorithm \\
&\verb+'maximum'+&specified in \verb+\algorithms\mark+\\
&\verb+'uniform'+&\\
\\
\verb+maxNrDoF+& a non &there will be no more \\
 &negative & refinement if this number of degrees of freedom \\
&integer &is reached\\
\\
\verb+problemType+&\verb+'elliptic'+ &defines the type of problem \\
 &\verb+'elasticity'+&\\
\\
\verb+solver+&\verb+'direct'+ &method how the resulting linear system \\
 (optional)&\verb+'multigrid'+&will be solved \\
&&(name of a file in \verb+\algorithms\linSysSolvers\+)\\
\\
\verb+refine+&\verb+'bisection'+ &method how to refine \\
 (optional)&\verb+'redGreenBlue'+&(name of a file in \verb+\algorithms\refine\+)\\
\\
\verb+estimate+&\verb+'estimate'+ &a string which represents an \\
 (optional)&&error estimating routine, \\
&& for future implementation of other\\
&& error estimators
\end{longtable}

\clearpage