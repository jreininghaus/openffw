\bigskip
\subsection{Predefined Problem Definitions}
\label{sect:ImplementedProblems_ProblemDefinition}

\noindent There are various problem definitions for elliptic PDEs and for linear elasticity already defined, for which we give an overview in the following.

\medskip

\noindent All problem definitions contain functions for the coefficients $\kappa$, $\lambda$, $\mu$, the external load $f$, and the boundary $u_D$ and $g$. Some of the problem-definition-filenames contain a suffix \path{_exact}. Here a function $u_{\text{exact}}$ which represents the exact solution is given and the data-functions $f$, $u_D$ and $g$ are computed automatically. With those problem definitions it is possible to compute the exact errors, e.g. energy error between $u_h$ and $u_{\text{exact}}$. Problem definitions for elasticity additionally contain standard parameters for the material parameter $\nu$ and $E$.

\bigskip

\noindent If you want to create your own problem definition, create an \code{.m}-file named \path{Elliptic_<problemName>} in \path{.\problems\elliptic} or \path{Elasticity_<problemName>} in (\path{.\problems\elasticity}), which contains all necessary information. For details concerning the structure of the problem definition, see Section \ref{sect:QuickStart}.

\bigskip

\subsubsection{Predefined Problem Definitions for Elliptic PDEs}$ $\\

\noindent\emph{Elliptic\_Lshape}

\smallskip

\noindent Model example $-\Delta u = 1$, $u_D=0$ and $g=0$ on an L-shaped domain.

\bigskip

\noindent\emph{Elliptic\_Lshape\_exact}

\smallskip

\noindent The exact solution $u(r,\phi)=r^{2/3}\sin(2/3\phi)$ in polar coordinates on an L-shaped domain with Neumann boundary is given. The corresponding right hand side $f$ is zero and induced boundary data. The coefficients of the elliptic PDE correspond to the Laplacian operator.
\bigskip

\noindent\emph{Elliptic\_Square}\smallskip\\
Model example $-\Delta u = 1$, $u_D=0$ and $g=0$ on a squared domain.
\bigskip

\noindent\emph{Elliptic\_Square\_exact}\smallskip\\
The exact solution $u(x,y) = x(1-x)\,y(1-y)$ as well as its gradient and the right-hand side $f$ are given with the PDE-coefficients $\kappa$, $\lambda$ and $\mu$ all zero, on a squared domain is given.
\bigskip

\noindent\emph{Elliptic\_SquareFullElliptic\_exact}\smallskip\\
The exact solution $u(x,y) = \sin(x^3)\cos(y^\pi)+x^8-y^9+x^6 y^{10}$ with the PDE-coefficients $\kappa$ being the identity, $\lambda = \left( \begin{smallmatrix} 5\sin(x+y)\\ 6\cos(x+y) \end{smallmatrix}\right)$ and $\mu=7$ on a squared domain is given. The symbolic toolbox of MATLAB computes all necessary information, i.e., $f$, $u_D$ and $g$.
\bigskip

\noindent\emph{Elliptic\_HexagonalSlit\_exact}\smallskip\\
The exact solution $u(r,\phi)=r^{1/4}\sin(1/4\phi)$ in polar coordinates with the load $f\equiv0$, the Dirichlet function $u_D=u|_{\Gamma_D}$, and coefficients belonging to the Laplacian are given on a slitted hexagon.
\bigskip

\noindent\emph{Elliptic\_Waterfall\_exact}\smallskip\\
The waterfall function $$u(x,y) = xy(1-x)(1-y) \arctan\left(k(\sqrt{(x-5/4)^2 + (y+1/4)^2}-1)\right)$$ is given. The parameter $k$ controls the slope of $u$. For $k\rightarrow\infty$ the slope of the function tends to infinity. This parameter is stored in the structure \code{p} at \code{p.PDE.k} and can be changed in the starting scripts. The domain is a square with homogeneous Dirichlet boundary. We look at the problem $-\Delta u = f$. The load $f$ is computed from $u$.
\bigskip

\noindent\emph{Elliptic\_Template}\smallskip\\
A predefined template for generating problem definitions. For given data $f, u_D$ and $g$ one can compute the corresponding discrete solution $u_h$.
\bigskip

\noindent\emph{Elliptic\_Template\_Exact}\smallskip\\
A predefined template for generating problem definitions. For a given function $u_{\text{exact}}(x,y)$ in cartesian coordinates and coefficients $\kappa,\lambda,\mu$, the symbolic toolbox of MATLAB computes all necessary information, i.e., $f$, $u_D$ and $g$.
\bigskip


\subsubsection{Predefined Problem Definitions for Elasticity}$ $\\

\noindent\emph{Elasticity\_Cooks}\smallskip\\
A tapered panel is clamped on one end and subjected to a surface load in
vertical direction on the opposite end with $f = 0$ and $g(x, y) = (0, 1000)$ if
$(x, y) \in  \Gamma_N$ with $x = 48$ and $g = 0$ on the remaining part of $\Gamma_N$, the Young modulo $E =
2900$, and the Possion ratio $\nu = 0.3$.
\bigskip


\noindent\emph{Elasticity\_Square\_exact}\smallskip\\
The unit square with the given function $u(x,y)=10^{-5}\left(\begin{smallmatrix} \cos((x+1)(y+1)^2) \\
\sin((x+1))\cos(y+1) \end{smallmatrix}\right)$. Young modulo and the Possion ratio are set to $E=10^5$ and $\nu = 0.3$.
\bigskip


\noindent\emph{Elasticity\_Square\_Neumann\_exact}\smallskip\\
Besides the geometry we have the same problem definition as in \emph{Elasticity\_Square}. The domain changes from \code{Square} to \code{SquareNeumann}.
\bigskip


\noindent\emph{Elasticity\_Lshape\_exact}\smallskip\\
Using polar coordinates $(r, \theta)$, $-\pi < \theta \leq \pi$ $u$ with radial component $u_r,u_\theta$ reads
$$
u_{r}(r,\theta) = \frac{r^{\alpha}}{2\mu}
    (-(\alpha+1)\cos((\alpha+1)\theta)+
           \nonumber     (C_{2}-
    (\alpha+1))C_{1}\cos((\alpha-1)\theta)),
$$
and
$$ u_{\theta}(r,\theta) = \frac{r^{\alpha}}{2\mu}
    ((\alpha+1)\sin((\alpha+1)\theta)+
          \nonumber     (C_{2}+\alpha-1)C_{1}\sin((\alpha-1)\theta)).
$$
The parameters are
$C_{1}=-\cos((\alpha+1)\omega)/\cos((\alpha-1)\omega)$, $C_{2} = 2(\lambda+2\mu)/(\lambda+\mu)$ where $\alpha = 0.54448...$ is the positive solution of  $\alpha \sin 2\omega + \sin 2\omega\alpha = 0$ for $ \omega= 3 \pi /4$; the Young modulus is $E=10^5$, Poisson ratio $\nu = 0.3$, and the volume force $f\equiv 0$.
\bigskip


\noindent\emph{Elasticity\_Template}\smallskip\\
A predefined template for generating problem definitions. For given data $f,u_D$ and $g$ and coefficients $\kappa,\lambda,\mu$ one can compute the corresponding discrete solution $u_h$.
\bigskip


\noindent\emph{Elasticity\_Exact\_Template}\smallskip\\
A predefined template for generating problem definitions. For a given function $u_{\text{exact}}(x,y)$ in cartesian coordinates, the symbolic toolbox of MATLAB computes all necessary information, i.e., $f$, $u_D$ and $g$. 