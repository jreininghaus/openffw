\subsection{Eigenwert Probleme}
Gesucht sind die Eigenwerte $\omega$ und Eigenfunktionen $u$, $\lVert u\rVert_{L^2}=1 $, des elliptischen Eigenwertproblems
\begin{align*}
  -div(\kappa(x)\cdot\nabla u) + \lambda(x)\cdot\nabla u + \mu(x)\cdot u &= \omega u \\
\nonumber
  u &= u_D \hbox{, auf } \Gamma_D \\
  \frac{\partial u}{\partial \eta} &= g \hbox{, auf } \Gamma_N\; .
\end{align*}
Damit das \FFW alle Einstellungen in Bezug auf Eigenwertprobleme
�bernimmt muss zuerst
\begin{verbatim}
p.params.problem.type = 'eigenvalue';
p.params.solver = 'eigenvalue';
\end{verbatim}
gesetzt werden. Dies kann auch mittels eines Aufrufs von
\begin{verbatim}
p = configureP(pdeSolver,problem,mark,maxNrDoF,[],'eigenvalue');
\end{verbatim}
geschehen.\\
In der Datei \path{startEigenvalue.m} wurden alle Einstellungen schon vorgenommen. Dort stehen auch ein paar Problemstellungen zur Auswahl.

\subsubsection{Lineares Gleichungssystem l�sen}
\hspace{0cm}

\medskip
\noindent
\textbf{Datei:} \path{algorithms/linSysSolvers/eigenvalue/solve.m}\\[1.5ex]
Durch Finite Element Discretisierung des Eigenwert Problems erh�lt man ein verallgemeinertes Eigenwertproblem f�r Matrizen.
\begin{equation*}
A u = \omega B u\;
\end{equation*}
Es wird die Funktion \code{eigs} von MATLAB benutzt die unter Anderem auf das Packet ARPACK zur�ckgreift. Der Parameter \code{p.params.curEigenvalue} gibt dabei an, welcher Eigenwert berechnet werden soll. \code{eigs} berechnet alle \code{curEigenvalue} kleinsten Eigenwerte und Eigenfunktionen. Da das Gitter aber nur f�r den einen Eigenwert bzw. Eigenfunktion adaptiv verfeinert werden kann, wird auch nur dieser gespeichert. Die Anzeige wird vorher auf 'stumm' gestellt.
\begin{verbatim}
options.disp = 0;
[V,D] = eigs(A(freeNodes,freeNodes),B(freeNodes,freeNodes),...
             curEigenvalue,'sm',options);
\end{verbatim}

\subsubsection{PDESolver}
Zur Zeit ist nur die P1-FEM implementiert. Diese befindet sich im Ordner \path{/PDESolvers/P1-Eigenvalue}.
\begin{verbatim}
p.params.pdeSolver = 'P1-Eigenvalue';
\end{verbatim}
Die Steifigkeitsmatrix A und die Massenmatrix B wird dabei in
\begin{verbatim}
p.level(end).A
p.level(end).B
\end{verbatim}
gespeichert.

\subsubsection{Problem}
Zwei Problemdefinitionen stehen aktuell zur Verf�gung. Zum einen das Laplace Problem auf dem Quadrat f�r homogene Randdaten und zum anderen das Laplace Problem auf dem L-Shape ebenfalls mit homogenen Randdaten. Diese befinden sich im Verzeichniss \path{/problems/eigenvalue/}
\begin{verbatim}
p.params.problem.name = 'Laplace-Square-exact';
p.params.problem.name = 'Laplace-Lshape';
\end{verbatim}

\subsubsection{Parameter}
\hspace{0cm}

\medskip \noindent
\textbf{p.params.curEigenvalue}\\
Gibt an welcher Eigenwert ausgehend vom kleinsten Eigenwert berechnet werden soll (Default = 1).

\medskip \noindent
\textbf{p.params.modules.mark.refineFirstLevel}\\
Um gr�ssere Eigenwerte berechnen zu k�nnen muss auch die Matrix gross genug sein. D.h. es m�ssen gen�gend innere Knoten vorhanden sein. Um das zu garantieren kann mit diesem Parameter angegeben werden wie oft das Gitter uniform verfeinert wird, bevor die erste L�sung berechnet wird (Default = 1).

\subsubsection{ErrorType}
Zus�tzlich zu den anderen Typen ('estimatedError', 'L2error', 'H1semiError') steht bei Eigenwert Problemen noch der Typ 'eigenvalueError' zur Verf�gung. So l�sst sich mittels
\begin{verbatim}
p = show('drawError_eigenvalueError',p);
\end{verbatim}
der exakte Fehler des Eigenwertes in einer Graphik mit logarithmischen Skalen darstellen. Dazu mus vorher der exakte Eigenwert definiert worden sein.
\begin{verbatim}
p.problem.eigenvalue_exact = exakter Eigenwert;
\end{verbatim}
