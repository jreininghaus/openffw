\section{Mesh Generation}
\label{sect:MeshGeneration}

\noindent The aim of adaptive finite elements is a problem-based generation of a triangulation. This triangulation is then optimal in the sense that it reflects the structure of the problem. Therefore there is the well known AFEM-Loop which consists of the following steps

\begin{align*}
\text{Mark }\Longrightarrow\text{ Refine }\Longrightarrow\text{ Solve }\Longrightarrow\text{Estimate}.
\end{align*}

\noindent In the following we will present a short introduction of the steps \textit{Mark} and \textit{Refine} which are the major steps in construction of triangulations. The \textit{Solve}-step is just solving the linear system of equations. Here we have taken the built-in MATLAB-solver. There is also an iterative solver for the $P_1$-FE, the so-called multigrid solver. This implementation is only a proof of concept and will not be explained further. For readers who are interested in more details we refer to \cite{Bra,HaFaIo}. Here the general concepts of iterative solvers are explained. The \textit{Estimate}-step, which is essential for the \textit{Mark}-step if one has no a~priori knowledge about the problem, consist of the estimation of the error of the discrete solution. A lot of work was done in this field. An exemplary collection of these works is \cite{Brandts,BrCC,CC1,CC2,CC2004,CCFu,CCFu2,CCHo1,CCHo2,Ver}. In the \FFW mostly the residual based error estimator is taken. Since a theoretical treatment of the various estimators is not the aim of this documentation we refer the reader to the literature in bibliography.


\subsection{Input: Assumptions on course triangulation $\T_0$.}
\label{sec:mesh:input}
The initial mesh $\T_0$ is a \emph{regular triangulation} of $\Omega\subset\mathbb{R}^n$ into closed triangles in the sense that two distinct closed-element domains are either disjoint or their intersection is one common vertex or one common edge. We suppose that each element with domain in $\T_0$ has at least one vertex in the interior of $\Omega$.\\
Given any $T\in\T_0$, one chooses one of its edges $E(T)$ as a \emph{reference edge} such that the following holds. An element $T\in\T_0$ is called isolated if $E(T)$ either belongs to the boundary or equals the side of another element $K\in T_0$ with $E(T)=\partial T\cap\partial K\neq E(K)$. Given a regular triangulation $\T_0$, Algorithm~2.1 of~\cite{CC2004} computes the reference edges ($E(T):T\in\T_0$) such that two distinct isolated triangles do not share an edge. This is important for the $H^1$ stability of the $L^2$ projection in subsection~\ref{sec:Mesh:Properties}.

\subsection{Mark}
In order to refine the mesh adaptively we first have to say where.
In difference to other implementations we generally mark only edges 
instead of elements. If we want to refine an element we therefor
have to mark all its edges.
An edge is refined depending on the local error indicators.
There are three different contributions to the total estimated error.
There is the local error on an edge $\eta_E$ (\code{etaEd}), on an element $\eta_T$ (\code{etaT}) and the 
local oscillations $\eta_{osc}$ (\code{etaOsc}). The oscillations on elements are refined \emph{bisec5} instead of
\emph{red}. To compute the set of marked edges $\M_\ell$ there are four different
strategies implemented.
%
%
\subsubsection{uniform} $ $\\
File: \path{.\algorithms\mark\uniform.m}\\[1.5ex]
All elements are refined uniformly \emph{red}. Therefore 
$\M_\ell = \mathcal{E}_\ell$ and no elements are marked for \emph{bisec5} refinement.
\begin{pcode}
refineEdges = true(nrEdges,1);
refineElemsBisec5 = false(nrElems,1);
\end{pcode}
%
\subsubsection{maximum} $ $\\
File: \path{.\algorithms\mark\maximum.m}\\[1.5ex]
The maximum algorithm defines the set $\M_\ell\subseteq\E_\ell$ of marked edges such that for all $E\in\M_\ell$
\begin{equation*}
\eta_{E} >  \theta \cdot \max_{K\in\E_\ell}\eta_{K},
\end{equation*}
where $\theta \in [0,1]$ is a constant (default: $\theta = 0.5$).
The MATLAB code is printed in the next line.
\begin{pcode}
refineEdges = (etaEd > thetaEd * max(etaEd));
\end{pcode}
\code{True} means that the edge is marked and \code{false} it
is not.
If elements should be refined all the edges of an element $T$ belong to the set $\M_\ell$ if
\begin{equation*}
\eta_{T} >  \theta \cdot \max_{K\in\T_\ell}\eta_{K}.
\end{equation*}
The code realizes this by first marking the elements and than marking all the edges of marked elements.
\begin{pcode}
I = (etaT > thetaT * max(etaT))';
refineElems(I) = true;
refineEdges4e = ed4e(refineElems,:);
refineEdges(refineEdges4e(:)) = true;
\end{pcode}
Elements that have large osscilations
(\code{etaOsc}) are treated analogously. The reason why they are refined with \emph{bisec5}
instead of \emph{red} is that \emph{bisec5} generates
a new node in the interior of the element.\bigskip

\noindent The maximum algorithm can be very inefficient. Suppose we have the
same error on each edge except of very few edges and on those edges the error
is very large. Then the maximum algorithm might only refine these few edges 
(cf. figure~\ref{sect:MeshGeneration.Mark.maximum.fig}).

\begin{figure}
\setlength{\unitlength}{6cm}
\begin{picture}(1, 1)
\qbezier(0.1, 0.2)(0.15,0.5)(0.2,0.2)
\qbezier(0.2, 0.2)(0.25,0.25)(0.3,0.2)
\qbezier(0.3, 0.2)(0.35,0.4)(0.4,0.2)
\qbezier(0.4,0.2)(0.45,0.5)(0.5, 0.3)
\qbezier(0.5,0.3)(0.525,0.6)(0.55, 1)
\qbezier(0.55, 1)(0.575,0.6)(0.6, 0.2)
\qbezier(0.6,0.2)(0.65,0.5)(0.7, 0.2)
\qbezier(0.7, 0.2)(0.75,0.4)(0.8,0.2)
\qbezier(0.8, 0.2)(0.85,0.25)(0.9,0.2)
\put(0.05,0.5){\line(1,0){0.9}}
\put(1,0.5){\tiny 0.5 max}
\put(0,0){\vector(0,1){1}}
\put(0,0){\vector(1,0){1}}
\put(1.05,0){\tiny edge or element}
\put(0.05,0.9){\tiny estimated error}
\end{picture}
\caption{The maximum algorithm might mark only a few edges.}
\label{sect:MeshGeneration.Mark.maximum.fig}
\end{figure}

\subsubsection{bulk}
The bulk algorithm defines the set $\M_\ell$ of marked edges such that
\begin{equation*}
 \sum_{E\in \M_\ell}\eta_E^2 \geq  \theta \cdot \sum_{K\in\E_\ell} \eta_K^2,
\end{equation*}
or it contains all the edges of elements $T\in \mathcal{K}_\ell$ that satisfy
\begin{equation*}
 \sum_{T\in\mathcal{K}_\ell}\eta_T^2 \geq  \theta \cdot \sum_{K\in\T_\ell} \eta_K^2,
\end{equation*}
It is important to see that the set  $\M_\ell$ of edges that
are selected by this condition is not unique.
Here we use an greedy approach. We iteratively take
those edges with the largest error. Using this approach 
guarantees that we have the smallest possible set of
marked edges that satisfy the bulk condition.
In the following the corresponding MATLAB lines are printed.
\begin{pcode}
[sortedEtaEd,I] = sort(etaEd,'descend');
sumEtaEd = cumsum(sortedEtaEd.^2);
k = find(sumEtaEd >= thetaEd * norm(etaEd,2)^2,1,'first');
[sortedEtaT,I] = sort(etaT,'descend');
sumEtaT = cumsum(sortedEtaT.^2);
k = find(sumEtaT >= thetaT * norm(etaT,2)^2,1,'first');
refineElems(I(1:k)) = true;
refineEdges4e = ed4e((refineElems | refineElemsBisec5),:);
refineEdges(refineEdges4e(:)) = true;
\end{pcode}
The problem that only a few edges or element are marked 
like with the maximum algorithm cannot happen here, see Figure~\ref{sect:MeshGeneration.Mark.bulk.fig}.

\begin{figure}
\setlength{\unitlength}{6cm}
\begin{picture}(1, 1)
\qbezier(0.1, 0.2)(0.15,0.5)(0.2,0.2)
\qbezier(0.2, 0.2)(0.25,0.25)(0.3,0.2)
\qbezier(0.3, 0.2)(0.35,0.4)(0.4,0.2)
\qbezier(0.4,0.2)(0.45,0.5)(0.5, 0.3)
\qbezier(0.5,0.3)(0.525,0.6)(0.55, 1)
\qbezier(0.55, 1)(0.575,0.6)(0.6, 0.2)
\qbezier(0.6,0.2)(0.65,0.5)(0.7, 0.2)
\qbezier(0.7, 0.2)(0.75,0.4)(0.8,0.2)
\qbezier(0.8, 0.2)(0.85,0.25)(0.9,0.2)
\put(0.05,0.25){\line(1,0){0.9}}
\put(1,0.25){\tiny bulk}
\put(0,0){\vector(0,1){1}}
\put(0,0){\vector(1,0){1}}
\put(1.05,0){\tiny edge or element}
\put(0.05,0.9){\tiny estimated error}
\end{picture}
\caption{With the bulk algorithm it cannot happen that only
         a few edges are marked.}
\label{sect:MeshGeneration.Mark.bulk.fig}
\end{figure}

\subsubsection{graded}
\textbf{file:} \path{.\algorithms\mark\graded.m}\\[1.5ex]
Refinement with graded grids is a a priori mesh refinement toward singular corner points.
In the past the so called $\beta$-graded grids were very popular. The a priori analysis consists of the following theorem.
%
\begin{theorem}
Let $\mathcal{T}$ be a regular triangulation of $\Omega=T_{ref}$ such that for given $N\in\mathbb{N}$ and $\beta>0$
\begin{enumerate}
\item $\mathcal{T}$ contains the element $T_0=conv\{(0,0),(N^{-\beta},0),(0,N^{-\beta})\}$.
\item For each $T\in \mathcal{T}\backslash\{T_0\}$ and all $x\in T$ one has $diam(T)\leq c\frac{1}{N}|x|^{1-\beta}$
\end{enumerate}
Then, if $\alpha + \beta > 2$, it follows that
\begin{equation*}
|| \nabla(u_{\alpha} - I_{\mathcal{T}}u_{\alpha})||_{L^2(\Omega)}\leq c N^{-min\{1,\alpha\beta\}}\; .
\end{equation*}
where $u_{\alpha}$ is the corner singularity function and $\alpha$ depends on the opening angle in the corner singularity.
\end{theorem}
\noindent
It is known that $\beta$-graded grids satisfy this conditions. But here we follow another approach. We simply refine our mesh with \emph{red}-\emph{green}-\emph{blue} refinement until these conditions are satisfied. This leads to simpler algorithms since $\beta$-graded grids are difficult to implement. Another advantage is that the angles of the triangulation only depend on the initial triangulation. The parameter $\beta$ can be modified by setting the value
\begin{pcode}
p.params.modules.mark.graded.beta
\end{pcode}
For the L-shaped domain the default value $1/3$ is optimal. It is important to say that the implemented graded algorithm
assumes that the problem has exactly one singularity located at the origin.


\subsection{Closure}
In the process of generating adaptive meshed you have to
be careful that the angles of the element are bounded 
due to the maximum angle condition. To guarantee this you
additionally have to refine all reference edges of elements
that have marked edges.
This is done in the function closure.

\medskip
\noindent
\textbf{file:} \path{.\algorithms\misc\closure.m}\\[1.5ex]
\begin{tabular}{@{} l l}
\multicolumn{2}{@{} l}{\textbf{function p = closure(p)}} \\
\textbf{Input:}  & p - ffw\\
\textbf{Output:} & p - ffw\\
\end{tabular}

\medskip
\noindent
We briefly say that although the following code
can have quadratic runtime it has lineare runtime
in average. From convergence theory we know that for
a sequence of triangulations the number of the additionally refined 
reference edges are linear in the number of levels.
Therefor we know that it is in average constant at each
level. Because of that the while loop will be called
in average for a constant number of times. 
\begin{pcode}
I =  refineEdges(ed4e(:,2)) | refineEdges(ed4e(:,3));
while nnz(refineEdges(refEd4e(I))) < nnz(I);
   refineEdges(refEd4e(I)) = true;
   I =  refineEdges(ed4e(:,2)) | refineEdges(ed4e(:,3));
end
\end{pcode}


\subsection{Refine}
%
Given a triangulation $\T_\ell$ on the level $\ell$, let $\E_\ell$ denote its set of interior edges and suppose that $E(T)$ ($E(T):T\in\T_\ell$) denotes the given reference edges. There is no need to label the reference edges $E(T)$ by some level $\ell$ because $E(T)$ will be the same edge of $T$ in all triangulations $\T_m$ which include $T$. However, once $T$ in $\T_\ell$ is refined, the reference edges will change too.
After the closure algorithm each element has either $k = 0,1,2$ or 3 of 
its edges marked for refinement and because of the closure 
algorithm the reference edge belongs to it if $k\geq 1$. 
Therefore, exactly one of the four refinement rules of Figure~\ref{f:3} is applied.
This specifies sub triangles and their reference edges in the new triangulation $\mathcal{T}_{\ell+1}$.
In general there are four different cases to refine an element. 
Elements with no marked edges are not refined, elements with one marked edge
are refined \emph{green}, elements with 
two marked edges are refined \emph{blue} and elements with tree 
marked edges are refined \emph{red}. Where \emph{blue} refinement
is divided into the two cases \emph{blueleft} and \emph{blueright}.\bigskip


\begin{figure}[!ht]
\begin{center}
\setlength{\unitlength}{2cm}
\begin{picture}(5,1.1)

%red
\put(0,0){\line(1,0){2}}
\put(0,0){\line(1,1){1}}
\put(2,0){\line(-1,1){1}}
\put(0.5,0.5){\line(1,0){1}}
\put(1,0){\line(-1,1){0.5}}
\put(1,0){\line(1,1){0.5}}
\put(0.5,0.5){\circle*{0.05}}
\put(1,0){\circle*{0.05}}
\put(1.5,0.5){\circle*{0.05}}
\put(0,0.75){\emph{red}}
\put(-0.1,-0.05){1}
\put(0.96,1.02){3}
\put(2.02,-0.05){2}
\put(0.8,-0.2){$new_1$}
\put(0,0.5){$new_3$}
\put(1.6,0.5){$new_2$}
\put(0.7,0.55){\line(1,0){0.6}}
\put(0.7,0.45){\line(1,0){0.6}}
\put(0.2,0.05){\line(1,0){0.6}}
\put(1.2,0.05){\line(1,0){0.6}}


%green
\put(3,0){\line(1,0){2}}
\put(3,0){\line(1,1){1}}
\put(5,0){\line(-1,1){1}}
\put(4,0){\line(0,1){1}}
\put(4,0){\circle*{0.05}}
\put(3,0.75){\emph{green}}
\put(2.9,-0.05){1}
\put(3.96,1.02){3}
\put(5.02,-0.05){2}
\put(3.8,-0.2){$new_1$}
\put(3.2,0.1){\line(1,1){0.6}}
\put(4.8,0.1){\line(-1,1){0.6}}
\end{picture}
\vspace{1cm}

\begin{picture}(5,1)

%blue left
\put(0,0){\line(1,0){2}}
\put(0,0){\line(1,1){1}}
\put(2,0){\line(-1,1){1}}
\put(1,0){\line(0,1){1}}
\put(1,0){\line(-1,1){0.5}}
\put(0.5,0.5){\circle*{0.05}}
\put(1,0){\circle*{0.05}}
\put(0,0.75){\emph{blue left}}
\put(-0.1,-0.05){1}
\put(0.96,1.02){3}
\put(2.02,-0.05){2}
\put(0.8,-0.2){$new_1$}
\put(0,0.5){$new_2$}
\put(1.8,0.1){\line(-1,1){0.6}}
\put(0.2,0.05){\line(1,0){0.6}}
\put(0.9,0.25){\line(0,1){0.5}}

%blue right
\put(3,0){\line(1,0){2}}
\put(3,0){\line(1,1){1}}
\put(5,0){\line(-1,1){1}}
\put(4,0){\line(0,1){1}}
\put(4,0){\line(1,1){0.5}}
\put(4.5,0.5){\circle*{0.05}}
\put(4,0){\circle*{0.05}}
\put(2.8,0.75){\emph{blue right}}
\put(2.90,-0.05){1}
\put(3.96,1.02){3}
\put(5.02,-0.05){2}
\put(3.8,-0.2){$new_1$}
\put(4.6,0.5){$new_2$}
\put(3.2,0.1){\line(1,1){0.6}}
\put(4.2,0.05){\line(1,0){0.6}}
\put(4.1,0.25){\line(0,1){0.5}}

\end{picture}
\end{center}

\caption{\label{f:3} \emph{Red}, \emph{green} and \emph{blue} refinement.
         The new reference edge is marked through a second line in parallel opposite the new vertices 
         $new_1$, $new_2$ or $new_3$.}
\end{figure}


\noindent
All elements that are marked in refineElemsBisec5 are refined \emph{bisec5} instead of
\emph{red}, see Figure~\ref{f:4}.\bigskip

\begin{figure}[!ht]
\setlength{\unitlength}{2cm}
\begin{center}
\begin{picture}(2,1.1)
%bisec5
\put(0,0){\line(1,0){2}}
\put(0,0){\line(1,1){1}}
\put(2,0){\line(-1,1){1}}
\put(0.5,0.5){\line(1,0){1}}
\put(1,0){\line(-1,1){0.5}}
\put(1,0){\line(1,1){0.5}}
\put(1,0){\line(0,1){1}}
\put(0.5,0.5){\circle*{0.05}}
\put(1,0){\circle*{0.05}}
\put(1.5,0.5){\circle*{0.05}}
\put(1,0.5){\circle*{0.05}}
\put(-1,0.75){\emph{bisec5}}
\put(-0.10,-0.05){1}
\put(0.96,1.02){3}
\put(2.02,-0.05){2}
\put(0.8,-0.2){$new_1$}
\put(0,0.5){$new_3$}
\put(1.6,0.5){$new_2$}
\put(1,0.55){$new_4$}
%\put(0.25,-0.1){\vector(0,1){0.1}}
%\put(0.75,-0.1){\vector(0,1){0.1}}
%\put(0.73,0.2){\vector(-2,1){0.1}}
%\put(0.27,0.2){\vector(2,1){0.1}}
%\put(0.74,0.8){\vector(-2,-1){0.1}}
%\put(0.26,0.8){\vector(2,-1){0.1}}
\put(0.65,0.55){\line(1,1){0.2}}
\put(1.35,0.55){\line(-1,1){0.2}}
\put(0.2,0.05){\line(1,0){0.6}}
\put(1.2,0.05){\line(1,0){0.6}}
\put(0.9,0.175){\line(-1,1){0.2}}
\put(1.1,0.175){\line(1,1){0.2}}
\end{picture}
\vspace{1ex}
\end{center}
\caption{\label{f:4} \emph{bisec5} refinement. The new reference edge is marked through a second line in parallel opposite the new vertices $new_1$, $new_2$, $new_3$ and $new_4$.}
\end{figure}

\noindent
The red-green-blue refinement is implemented in\\
File: \path{.\algorithms\refine\redGreenBlue.m}
\begin{pcode}
function p = redGreenBlue(p) 
% input:   p - FFW
% output:  p - FFW
\end{pcode}
%
At first we create the new c4n. Therefore we take the old c4n 
and add the new coordinates at the end of the list. The new 
coordinates are the midpoints of the marked edges.
%
\begin{pcode}
% Create new node numbers from refineEdges
newNode4ed = zeros(1,nrEdges);
newNode4ed( find(refineEdges) ) = ...
     (nrNodes+1):(nrNodes+nnz(refineEdges));
% Create coordinates of the new nodes
[dontUse,J,S] = find(newNode4ed);
c4n(S,:) = midPoint4ed(J,:);
\end{pcode}
%
In the next step the new n4e is build. At first we calculate the 
number of marked edges for each element. All the Elements that will 
not be refined, e.g. have no marked edge, are fist copied to the new n4e.
In the following new elements will be appended to the list.
%
\begin{pcode}
newNode4e = newNode4ed(ed4e);
unrefinedElems = find( all(newNode4e == 0 ,2) );
refineElems = find( any(newNode4e,2) );
nrMarkedEd4MarkedElems = sum(refineEdges( ed4e(refineElems,:) ),2);
newn4e = n4e(unrefinedElems,:);
\end{pcode}


\noindent
All elements that are to be \emph{red} or \emph{green} refined can be refined simultaneously.
In the case of \emph{green} refinement, it is important to know that the first edge of an element is 
always the reference edge, therefore the marked edge, and that all elements have math. positive 
orientation, i.e. counter clockwise. Therefore there is only one
way to refine an element \emph{green}. 
Instead we have two different cases with \emph{blue} refinement. 
Therefore we distinguish between \emph{blueleft} and \emph{blueright}
refinement. Again there is only one way to perform \emph{red} refinement.
For example the MATLAB code for \emph{green} refinement is
printed below. The new elements are 
$T_1=\conv(2,3,new_1)$ and $T_1=\conv(3,1,new_1)$.
The new reference edge are the edges between the nodes
$2,3$ and $3,1$.
\begin{pcode}
I = find(nrMarkedEd4MarkedElems == 1);
if ~isempty(I)
  gElems = refineElems(I);
  [dontUse,dontUse,newN] = find( newNode4e(gElems,:)' );
  newGreenElems = [n4e(gElems,[2 3]) newN;...
                   n4e(gElems,[3 1]) newN];
  newn4e = [newn4e;newGreenElems];
end
\end{pcode}
%
At the end the lists for the boundary, Db and Nb, are
updated.
\begin{pcode}
Db = updateBoundary(Db,DbEd,newNode4ed);
Nb = updateBoundary(Nb,NbEd,newNode4ed);
...
function newBoundary = updateBoundary(oldB,ed4b,newNode4ed)
if(isempty(oldB))
    newBoundary = [];
else
  unrefinedEd = find(~newNode4ed(ed4b));
  refineEd = find(newNode4ed(ed4b));
  newBoundary = [oldB(unrefinedEd,:);...
    oldB(refineEd,1) newNode4ed(ed4b(refineEd))' ;...
    newNode4ed(ed4b(refineEd))' oldB(refineEd,2)];
end
\end{pcode}

\subsection{Properties of the meshes}
\label{sec:Mesh:Properties}
This subsection lists a few results on the triangulation $\T_\ell$ obtained by \emph{red}-\emph{green}-\emph{blue} refinement under the assumptions on $\T_0$ of subsection~\ref{sec:mesh:input}. The non-elementary proofs can be found in~\cite{CC2004}.\\[1.5ex]
\indent\textbf{(i) } $\T_\ell$ is a regular triangulation of $\Omega$ into triangles; for each $T\in\T_\ell $ there exists one reference edge $E(T)$ which depends only on $T$ but not on the level $\ell$.\\
\indent\textbf{(ii) } For each $K\in\T_0$, $\T_\ell|_K:=\{T\in\T_\ell\,|\,T\subseteq K\}$ is the picture under an affine map $\Phi: K\to T_{ref}$ onto the reference triangle $T_{ref}=\textrm{conv}\{(0,0),(0,1),(1,0)\}$ by $\Phi(E(K))=\textrm{conv}\{(0,0),(1,0)\}$ and $\det D\Phi>0$. The triangulation $\widehat{T}_K :=\{\Phi(T): T\in\T,T\subseteq K\}$ of $K$ consists of right isosceles triangles. (A right isosceles triangle results from a square halved along a diagonal.)\\
\indent\textbf{(iii) }
The $L^2$ projection onto $V_\ell$ is $H^1$ stable, in the sense that 
$V_\ell:=\mathcal{P}_1(\T_\ell)$ denotes the piecewise affine space, i.e.
\begin{align*}
\mathcal{P}_1(\T_\ell \mathbb{R}^m) &:= \left\{ v\in C^\infty(T;\mathbb{R}^m): v\textrm{ affine on }T \right\},\\
\mathcal{P}_1(\T_\ell \mathbb{R}^m) &:= \left\{ v\in L^\infty(\Omega;\mathbb{R}^m): \forall T\in\T_\ell, v|_T\in \mathcal{P}_1(T;\mathbb{R}^m) \right\}.
\end{align*}
For any $v\in H_0^1(\Omega)$ the $L^2$ projection $\Pi v$ on $V_\ell$ satisfies
\begin{equation*}
\norm{\nabla\Pi v}{L^2(\Omega)} \leq C_1\norm{\nabla v}{L^2(\Omega)}\textrm{ and }
\norm{h^{-1}_\T\Pi v}{L^2(\Omega)} \leq C_2\norm{h^{-1}_\T v}{L^2(\Omega)}.
\end{equation*}
The constants $C_1$ and $C_2$ exclusively depend on $\T_0$.\\
\indent\textbf{(iv) } Approximation property of the $L^2$ projection
\begin{align*}
\sum_{T\in\T_\ell} \norm{h_{T}^{-1}(v-\Pi v)}{L^2(T)}^2+\sum_{E\in\E_\ell}\norm{h_{E}^{-1/2}(v-\Pi v)}{L^2(E)}^2 \leq C_3 \norm{\nabla v}{L^2(\Omega)}^2.
\end{align*}
